\chapter{Instruções}
Neste capítulo vamos mostrar como utilizar a nossa aplicação, os comandos estão listados em ordem, portanto a execução deles resulta no funcionamento do projeto. Na sequência, teremos a instalação das ferramentas necessárias no Linux e no Windows e posteriormente como será o procedimento para rodar a aplicação.

\section{Linux}
\subsection{Instalação do Node}
\begin{minted}[autogobble,breaklines,linenos,frame=lines,fontsize=\footnotesize]{bash}
	sudo apt-get install nodejs #Distros baseadas em Debian ou
	sudo pacman -S nodejs       #Distros baseadas em Arch Linux
\end{minted}

\subsection{Instalação do \textsc{npm}}
\begin{minted}[autogobble,breaklines,linenos,frame=lines,fontsize=\footnotesize]{bash}
	sudo apt-get install npm #Distros baseadas em Debian ou
	sudo pacman -S npm       #Distros baseadas em Arch Linux
\end{minted}

\subsection{Instalação do \emph{Create React App}}
\begin{minted}[autogobble,breaklines,linenos,frame=lines,fontsize=\footnotesize]{bash}
	sudo npm install -g create-react-app
\end{minted}

\section{Windows}
\subsection{Instalação do Node}
Pode ser baixado através do \emph{site} oficial:
\href{https://nodejs.org/en/download/}{https://nodejs.org/en/download/}

\subsection{Instalação do \textsc{npm}}
Como o NPM vem do Node e é instalado por padrão, não é preciso instalá-lo.

\subsection{Instalação do \emph{Create React App}}
\begin{minted}[autogobble,breaklines,linenos,frame=lines,fontsize=\footnotesize]{bash}
	npm install -g create-react-app
\end{minted}

\section{Instalação do \emph{CouchDB}}
Cada distribuição linux tem a instalação do \emph{CouchDB} de uma forma, podendo também baixar direto do \href{https://couchdb.apache.org}{site}, assim como feito no Windows. No GNU/Linux é importante verificar se o serviço/\emph{daemon} do banco de dados está funcionando, não colocaremos código para exemplificar, pois pode mudar de distribuição para distribuição.

Para verificar se o banco de dados está funcionando, basta entrar no endereço \href{http://localhost:5984/}{http://localhost:5984/}.

Para o funcionamento da aplicação, precisamos ativar as permossões de acesso de domínios, para acessar a \textsc{api} do banco basta acessar \href{http://localhost:5984/\_utils}{http://localhost:5984/\_utils}, com isso basta ir nas configurações do \emph{CouchDB}, na opção \textsc{cors} e clicar em \emph{Enable \textsc{cors}} e selecionar a opção \emph{All domains}.

\subsection{Alimentando o banco de dados}
Para colocar os conteúdos que temos no banco de dados, temos que utilizar os arquivos contidos na pasta \texttt{database/} da raiz do projeto, então vamos para a pasta:

\begin{minted}[autogobble,breaklines,linenos,frame=lines,fontsize=\footnotesize]{bash}
	cd database
\end{minted}

Antes de adicionar o conteúdo dos \texttt{.json}, precisamos criar o banco de dados para depois inserir as tuplas, então precisamos rodar os seguintes comandos:

\begin{minted}[autogobble,breaklines,linenos,frame=lines,fontsize=\footnotesize]{bash}
	curl -X PUT http://localhost:5984/petshop
	curl -X PUT http://localhost:5984/pets
	curl -X PUT http://localhost:5984/usuarios
	curl -X PUT http://localhost:5984/servicos
\end{minted}

E, agora para inserir os dados no banco de dados:

\begin{minted}[autogobble,breaklines,linenos,frame=lines,fontsize=\footnotesize]{bash}
	curl -d @petshop.json -H "Content-type: application/json" -X POST http://localhost:5984/petshop/_bulk_docs
	curl -d @pets.json -H "Content-type: application/json" -X POST http://localhost:5984/pets/_bulk_docs
	curl -d @usuarios.json -H "Content-type: application/json" -X POST http://localhost:5984/usuarios/_bulk_docs
	curl -d @servicos.json -H "Content-type: application/json" -X POST http://localhost:5984/servicos/_bulk_docs
\end{minted}

Com estes comandos temos todas as ferramentas necesssárias para o funcionamento
da nossa aplicação, tanto no Windows como no Linux.

\section{Rodando a aplicação}
Agora precisamos que entre na pasta do projeto pelo terminal.

\begin{minted}[autogobble,breaklines,linenos,frame=lines,fontsize=\footnotesize]{bash}
	cd ReactPetShop #Tanto Windows quanto Linux
\end{minted}

Rode o projeto instalando suas dependências:

\begin{minted}[autogobble,breaklines,linenos,frame=lines,fontsize=\footnotesize]{bash}
	npm install && npm start      #Windows
	npm install && npm start      #Linux
\end{minted}

Observação: caso, não consiga rodar o comando \texttt{npm install}, exclua a pasta \texttt{node\_modules} da raiz do projeto e tente novamente os comandos anteriores.

O projeto é aberto no navegador padrão assim que tudo estiver pronto, caso isso
não ocorra, é possível acessá-lo através do \emph{link}:
\href{http://localhost:3000/}{http://localhost:3000/}


























