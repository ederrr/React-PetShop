\chapter{Estrutura do \emph{site}}
Como o nosso \emph{site} é baseado praticamente em componentização, neste
capítulo passaremos uma noção de como o estruturamos.

\section{\textsc{app}}
No começo de tudo, temos a nossa aplicação, mais conhecida como \textsc{app},
componente a qual consiste o nosso \emph{site} inteiro e que será responsável
por formar nosso único arquivo \texttt{.html}, chamado \texttt{index.html}. A
\textsc{app} está separada em 4 partes, as quais são o \textbf{Topo},
\emph{\textbf{Navbar}}, \textbf{Conteúdo} e \textbf{Rodapé};

\subsection{Topo}
Neste componente nós temos três itens basicamente, o logo da aplicação, o qual
também é uma ligação para a página inicial, uma barra de busca no \emph{site} e
uma área de acesso ao usuário. O \textbf{Topo} é o único componente, dos quatro
principais, que tem uma parte dinâmica, além do componente \textbf{Conteúdo},
pois ela contém a parte do \emph{login}, onde o usuário pode entrar com o seu
\emph{e-mail} e senha ou, se já tiver entrado, acessar sua conta.

\subsubsection{Barra de busca}
\label{sub:barra}
Área de pesquisa que serve para buscar algum produto que tenha na loja. A parte
da busca ainda não está funcionando completamente, pois busca só palavras
completas na parte de produtos com \emph{case sensitive}, ou seja, se tiver um
produto com o nome “Produto 1” e procurarmos com as palavras “prod”, “produto
1”, “Prod” ou “Produto” não haverá retorno, só teremos sucesso se procurarmos
utilizando a palavra “Produto 1”.

\subsubsection{\emph{Login}}
Este componente, como foi falado anteriormente, serve para o usuário entrar na
sua conta ou se cadastrar, caso não tiver uma conta cadastrada. Caso
“\emph{logado}”, o usuário pode acessar a área de usuário, sair da sessão e
acessar o carrinho de compras. Um ponto importante de se falar é que na parte do
\emph{login} utilizamos um padrão de projeto (\emph{pattern design}) chamado
\emph{flux/redux}, responsável por criar uma store com as variáveis que
precisamos acessar por outras componentes.

\subsection{\emph{Navbar}}
Um menu que serve para acessar de forma mais intuitiva algumas partes do
\emph{site} como \textbf{Início}, o qual você consegue acessar de qualquer lugar
do \emph{site} a página inicial, \textbf{Serviços}, onde você consegue de
qualquer lugar da aplicação ir para a página de serviços que estão disponíveis,
e alguns tipos de animais, como \textbf{Cachorros}, \textbf{Gatos}, entre
outros, que são um atalho para os produtos da respectiva espécie.

\subsection{Conteúdo}
Além da parte do \textbf{Topo}, a qual tem a área de \emph{login}, essa é a
única parte do nosso \emph{site} que se muda, nela terá todas as informações não
estáticas do site. Um exemplo é na tela incial, no componente \textbf{Conteúdo}
temos uma área que fica um \emph{slide} de imagens mostrando os destaques que o
dono da aplicação pode colocar e, logo abaixo, temos listado alguns produtos de
cada espécie, esses produtos, na nossa aplicação está pegando os quatro
primeiros de cada espécie que estão cadastrados. Nada impede que coloquemos os
produtos de cada espécie que estão em promoção.

Nesta parte do site, será mostrada todas as informações de usuário, de serviços,
listagem de produtos, resultados de pesquisa, cadastros em geral, entre outros,
ou seja, este componente é responsável por chamar outros componentes.

\subsection{Rodapé}
No rodapé, há o nome completo de todos os integrantes do grupo, onde cada nome é
um \emph{link} que redireciona o usuário para o a conta do GitHub de cada um.
