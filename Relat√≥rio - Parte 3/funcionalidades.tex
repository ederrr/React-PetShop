\chapter{Funcionalidades}
\section{Página de \emph{index}}
Na página inicial o logo da aplicação sendo exibido corretamente e, de forma global, redireciona o usuário para o \emph{index} ao clicar.

A \emph{navbar} está funcional funcional, sendo possível acessar qualquer categoria desse menu. Na área de usuário, quando o usuário está “logado”, as opções de perfil de usuário, carrinho e sair são acessíveis. Já na barra de busca, temos uma busca funcional para buscas exatas, conforme descrito no item~\ref{sub:barra}.

Todos produtos, de todas as categorias, estão sendo exibidos com suas respectivas informações e o produto listado é acessível para área do produto, mostrando suas devidas informações.

\section{Página de produto}
Exibe o produto selecionado corretamente, permite a mudança das imagens armazenadas no banco e exibe suas informações contidas no banco.

O botão comprar adiciona o produto no carrinho e já redireciona o usuário para a página do carrinho, se tiver feito o \emph{login}, caso não tiver com a sua conta “logada”, ele redireciona para o carrinho, mas avisa que precisa estar “logado” para vizualizá-lo. Obs: ao logar, seu carrinho será exibido com o produto selecionado previamente.

\section{Página do carrinho}
A página exibe os produtos que estão no carrinho, se caso tiver produtos. Permite mudar a quantidade de cada produto, permite exclusão dos produtos do carrinho, porém quando uma exclusão é feita, não é feito a atualização automática da área do carrinho.

O botão fechar carrinho, finaliza uma compra, removendo todos os produtos do carrinho e redireciona o usuário para  a página principal. No entanto a compra não é salva em banco.

\section{Página de serviço}
Todos os serviços são exibidos corretamente com todas as suas informações trazidas do banco e cada serviço tem o seu botão agendar funcionando que, ao clicar, redireciona o usuário para a página de agendamento do serviço selecionado, se caso o usuário estiver “logado”.

\section{Página de agendamento}
O serviço que foi selecionado previamente é exibido com todas as suas informações. O usuário “logado” é reconhecido e seus \emph{pets} são listados, permitindo seleção do \emph{pet} para agendamento.

A escolha de data e hora está funcional, porém sem restrições. Já o valor do serviço é exibido, porém o agendamento não é registrado em banco ao clicar em agendar e o usuário é redirecionado para a página inicial.

\section{Página de busca}
A página de busca é acessada através de uma busca realizada na barra de busca, mostrando os produtos que foram encontrados pela ação realizada. Os produtos são exibidos corretamente.

Um subtítulo da página exbibe corretamente o que foi buscado e todos os produtos com este nome são exibidos e podendo ser acessados clicando em sua miniatura.

\section{Página de usuário}
Ao entrar com seu usuário e senha, a área de \emph{login} é modificada, permitindo o acesso ao perfil do usuário. As informações de usuário são buscadas e exibidas corretamente, com todos os seus \emph{pets}, que são listados, podendo ser selecionado para acessar suas informações no perfil do \emph{pet} escolhido.

Caso este usuário seja um administrador, uma área do administrador é exibida, caso contrário apenas são listadas áreas de compras e serviços realizados. Se o usuário for um administrador, ele pode acessar as áreas de inserção de produtos, serviços e usuários.

Compras e serviços realizados não são exibidos, assim como a área do administrador não é funcional. Ao clicar no botão de adicionar \emph{pet}, o usuário é redirecionado para um formulário de cadastro do \emph{pet}.

\section{Páginas de cadastro}
As páginas de cadastro de produtos, usuários e serviços são acessadas somente por um administrador, já o cadastro de \emph{pets} pode ser feito por qualquer tipo de usuário.

As páginas de cadastro estão com todos os campos funcionais, porém o botão cadastrar não tem funcionalidade.























