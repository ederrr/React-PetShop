%%%%%%%%%%%%%%%%%%%%%%%%%%%%%%%%%%%%%%%%%%%%%%%%%%%%%%%%%%%
%			Projeto 3 - CouchDB
%
%	Introdução ao Desenvolvimento Web			SCC0219
%	Prof. Dilvan de Abreu Moreira
%
%	Eder Rosati Ribeiro							8122585
%	Leonardo de Almeida Lima Zanguetin			8531866
%	Victor Luiz da Silva Mariano Pereira		8602444
%
%%%%%%%%%%%%%%%%%%%%%%%%%%%%%%%%%%%%%%%%%%%%%%%%%%%%%%%%%%%

% ---
% Definição do documento
% ---
\documentclass[
	% -- opções da classe memoir --
	12pt,				% tamanho da fonte
	openright,			% capítulos começam em pág ímpar
						% (insere página vazia caso preciso)
	%openany,			% capítulos começam em qualquer página
	twoside,			% para impressão em recto e verso. Oposto a oneside
	%oneside,			% para impressão somente em recto
	a4paper,			% tamanho do papel.
]{abntex2}
% ---

% ---
% Pacotes básicos
% ---
\usepackage{fontspec}
\usepackage{polyglossia}
	\setdefaultlanguage{portuges}
\usepackage{lmodern}			% Usa a fonte Latin Modern
\usepackage{indentfirst}		% Indenta o primeiro parágrafo de cada seção.
\usepackage{color}				% Controle das cores
\usepackage{graphicx}			% Inclusão de gráficos
	\graphicspath{{Figuras/}}
\usepackage{microtype}			% para melhorias de justificação
\usepackage{amssymb}
% ---

% ---
% Pacotes de citações
% ---
\usepackage[brazilian,hyperpageref]{backref}	% Paginas com as citações na
												% bibliografia
\usepackage[alf]{abntex2cite}	% Citações padrão ABNT
\usepackage{caption}			% Pacote para resolver o problema do
								% "listings", "minted" e códigos muito grandes
\usepackage{listings}			% Pacote para listagens de códigos
\usepackage{minted}				% Inserção de códigos fonte
% ---

% ---
% Pacotes para tabelas
% ---
\usepackage{longtable}
\usepackage{booktabs}
\usepackage{array}
% ---

% ---
% Pacotes adicionais
% ---
%\usepackage{showframe}			% Pacote para mostrar as caixas de texto
% ---

% ---
% Informações de dados para CAPA e FOLHA DE ROSTO
% ---
\titulo{Projeto 3 \\ \emph{CouchDB} PetShop}
\autor{
		Eder Rosati Ribeiro						---	8122585 \and\\
		Leonardo de Almeida Lima Zanguetin		---	8531866	\and\\
		Victor Luiz da Silva Mariano Pereira	---	8602444 \\
}
\local{Brasil}
\data{2018}
\instituicao{%
  Universidade de São Paulo -- USP
  \par
  Instituto de Ciências Matemáticas e de Computação -- ICMC
  \par
  Introdução ao Desenvolvimento Web -- SCC0219}
\tipotrabalho{Trabalho Acadêmico}
% ---

% ---
% Configurações de aparência do PDF final

% alterando o aspecto da cor azul
\definecolor{blue}{RGB}{41,5,195}

% informações do PDF
\makeatletter
\hypersetup{
	%pagebackref=true,
	pdftitle={\@title},
	pdfauthor={\@author},
	pdfsubject={CouchDB PetShop},
	pdfcreator={LuaLaTeX with abnTeX2},
	pdfkeywords={USP }{ICMC }{Introdução ao Desenvolvimento Web }{CouchDB }{PetShop},
	colorlinks=true,			% false: boxed links; true: colored links
	linkcolor=blue,				% color of internal links
	citecolor=blue,				% color of links to bibliography
	filecolor=magenta,			% color of file links
	urlcolor=blue,
	bookmarksdepth=4
}
\makeatother
% ---

% ---
% Espaçamentos entre linhas e parágrafos
% ---

% O tamanho do parágrafo é dado por:
\setlength{\parindent}{1.3cm}

% Controle do espaçamento entre um parágrafo e outro:
\setlength{\parskip}{0.2cm}  % tente também \onelineskip

% ---
% compila o indice
% ---
\makeindex
% ---

% ----
% Início do documento
% ----
\begin{document}

% Retira espaço extra obsoleto entre as frases.
\frenchspacing

% ----------------------------------------------------------
% ELEMENTOS PRÉ-TEXTUAIS
% ----------------------------------------------------------
\pretextual

% ---
% Capa
% ---
\imprimircapa
% ---

% ---
% Folha de rosto
% (o * indica que haverá a ficha bibliográfica)
% ---
\imprimirfolhaderosto*
% ---

% ---
% RESUMOS
% ---
% resumo em português
%\setlength{\absparsep}{18pt} % ajusta o espaçamento dos parágrafos do resumo
%\begin{resumo}
%\end{resumo}

% ---
% inserir lista de ilustrações
% ---
%\pdfbookmark[0]{\listfigurename}{lof}
%\listoffigures*
%\clearpage
% ---

% ---
% inserir lista de tabelas
% ---
%\pdfbookmark[0]{\listtablename}{lot}
%\listoftables*
%\clearpage
% ---

% ---
% inserir lista de códigos fonte
% ---
%\renewcommand\listoflistingscaption{Lista de Códigos Fonte}
%\listoflistings
%\clearpage
% ---

% ---
% inserir o sumario
% ---
%\pdfbookmark[0]{\contentsname}{toc}
%\tableofcontents*
%\clearpage
% ---

% ----------------------------------------------------------
% ELEMENTOS TEXTUAIS
% ----------------------------------------------------------
\textual

% ----------------------------------------------------------
% Corpo do trabalho
% ----------------------------------------------------------
\chapter*[Introdução]{Introdução}
Nesta terceira fase, a  final, da aplicação PetShop, nos foi solicitado que
concluíssemos o desenvolvimento de uma página \emph{web} utilizando as
linguagens \textsc{html5}, \textsc{css3}, as quais havíamos utilizado para
desenvolver a primeira parte do projeto, JavaScript e o uso de \emph{frameworks}, vistos na segunda parte, e agora a utilização de um banco de dados.

O objetivo desta parte é o refinamento de uma página no estilo
\emph{Single-Page Application} de um PetShop. Por termos feito já o
\emph{mockup} na primeira parte do trabalho e a utilização de JavaScrip com \emph{frameworks} da segunda parte, acreditamos que será mais simples de fazer essa terceira, e última, parte.

Na segunda parte do trabalho, utilizamos uma \textsc{api} para simular um banco de dados, pois pensamos que seria mais simples a migração para um banco que utiliza a arquitetura \textsc{rest}ful. Escolhemos para essa parte o \emph{CouchDB}, um banco de dados no\textsc{sql} que utiliza como linguagem JavaScrip e protocolo de comunicação o \textsc{http}.

\chapter{Funcionamento}
Como foi dito anteriormente, todo o \emph{site} foi desenvolvido utilizando o
\emph{React}, uma biblioteca responsável apenas pelo desenvolvimento do
\emph{front-end} em \emph{Single Page Application} (\textsc{spa}). Tudo que
julgamos conveniente à reutilização foi componentizado para reaproveitamento,
tanto nos itens que precisavam ser reutilizáveis quanto nas partes que tinham
papéis diferentes dentro do \emph{site}.

Essa biblioteca JavaScript roda através do Node, portanto é necessário a
instalação do Node e do \textsc{npm} (do inglês, \emph{Node Package Manager})
para iniciar um projeto. Com o \textsc{npm} é possível baixar e instalar o
\emph{Create React App}, um pacote responsável por configurar várias tecnologias
e trazer um projeto pronto para o desenvolvimento com todas as ferramentas
iniciais mais utilizadas no \emph{React}. Uma dessas tecnologias é o \textsc{jsx}, capaz de transformar os códigos JavaScript em códigos parecidos com \textsc{html5}.

Pode haver estranhamento com o código desenvolvido por aparentar utilizar
\textsc{html} misturado com JavaScript, porém, é tudo JavaScript e o Babel nos
auxilia nessa etapa. Todo o código pode ser realizado em \textsc{ecmas}crip 6 pois é transpilado e convertido em \textsc{ecmas}cript 5 para funcionar nos navegadores atuais que ainda não o suportam.

Juntamente com a parte estrutural do \emph{site}, também adicionamos uma base de dados, a qual alimenta a aplicação com o os dados previamente gerados pelo grupo. Em outras palavras, o banco de dados disponibiliza os dados por meio de uma \textsc{api} e o projeto mostra essas informações na tela para o usuário.

\chapter{Estrutura do \emph{site}}
Como o nosso \emph{site} é baseado praticamente em componentização, neste
capítulo passaremos uma noção de como o estruturamos.

\section{\emph{APP}}
No começo de tudo, temos a nossa aplicação, mais conhecida como \emph{APP},
componente a qual consiste o nosso \emph{site} inteiro e que será responsável
por formar nosso único arquivo \texttt{.html}, chamado \texttt{index.html}. A
\emph{APP} está separada em 4 partes, as quais são o \textbf{Topo},
\emph{\textbf{Navbar}}, \textbf{Conteúdo} e \textbf{Rodapé};

\subsection{Topo}
Neste componente nós temos três itens basicamente, o logo da aplicação, o qual
também é uma ligação para a página inicial, uma barra de busca no \emph{site} e
uma área de acesso ao usuário. O \textbf{Topo} é o único componente, dos quatro
principais, que tem uma parte dinâmica, além do componente \textbf{Conteúdo},
pois ela contém a parte do \emph{login}, onde o usuário pode entrar com o seu
\emph{e-mail} e senha ou, se já tiver entrado, acessar sua conta.

\subsubsection{Barra de busca}
Área de pesquisa que serve para buscar algum produto que tenha na loja. A parte
da busca ainda não está funcionando completamente, pois busca só palavras
completas na parte de produtos com \emph{case sensitive}, ou seja, se tiver um
produto com o nome “Produto 1” e procurarmos com as palavras “prod”, “produto
1”, “Prod” ou “Produto” não haverá retorno, só teremos sucesso se procurarmos
utilizando a palavra “Produto 1”.

\subsubsection{\emph{Login}}
Este componente, como foi falado anteriormente, serve para o usuário entrar na
sua conta ou se cadastrar, caso não tiver uma conta cadastrada. Caso
“\emph{logado}”, o usuário pode acessar a área de usuário, sair da sessão e
acessar o carrinho de compras. Um ponto importante de se falar é que na parte do
\emph{login} utilizamos um padrão de projeto (\emph{pattern design}) chamado
\emph{flux/redux}, responsável por criar uma store com as variáveis que
precisamos acessar por outras componentes.

\subsection{\emph{Navbar}}
Um menu que serve para acessar de forma mais intuitiva algumas partes do
\emph{site} como \textbf{Início}, o qual você consegue acessar de qualquer lugar
do \emph{site} a página inicial, \textbf{Serviços}, onde você consegue de
qualquer lugar da aplicação ir para a página de serviços que estão disponíveis,
e alguns tipos de animais, como \textbf{Cachorros}, \textbf{Gatos}, entre
outros, que são um atalho para os produtos da respectiva espécie.

\subsection{Conteúdo}
Além da parte do \textbf{Topo}, a qual tem a área de \emph{login}, essa é a
única parte do nosso \emph{site} que se muda, nela terá todas as informações não
estáticas do site. Um exemplo é na tela incial, no componente \textbf{Conteúdo}
temos uma área que fica um \emph{slide} de imagens mostrando os destaques que o
dono da aplicação pode colocar e, logo abaixo, temos listado alguns produtos de
cada espécie, esses produtos, na nossa aplicação está pegando os quatro
primeiros de cada espécie que estão cadastrados. Nada impede que coloquemos os
produtos de cada espécie que estão em promoção.

Nesta parte do site, será mostrada todas as informações de usuário, de serviços,
listagem de produtos, resultados de pesquisa, cadastros em geral, entre outros,
ou seja, este componente é responsável por chamar outros componentes.

\subsection{Rodapé}
No rodapé, há o nome completo de todos os integrantes do grupo, onde cada nome é
um \emph{link} que redireciona o usuário para o a conta do GitHub de cada um.

\chapter{Instruções}
Neste capítulo vamos mostrar como utilizar a nossa aplicação, os comandos estão listados em ordem, portanto a execução deles resulta no funcionamento do projeto. Na sequência, teremos a instalação das ferramentas necessárias no Linux e no Windows e posteriormente como será o procedimento para rodar a aplicação.

\section{Linux}
\subsection{Instalação do Node}
\begin{minted}[autogobble,breaklines,linenos,frame=lines,fontsize=\footnotesize]{bash}
	sudo apt-get install nodejs #Distros baseadas em Debian ou
	sudo pacman -S nodejs       #Distros baseadas em Arch Linux
\end{minted}

\subsection{Instalação do \textsc{npm}}
\begin{minted}[autogobble,breaklines,linenos,frame=lines,fontsize=\footnotesize]{bash}
	sudo apt-get install npm #Distros baseadas em Debian ou
	sudo pacman -S npm       #Distros baseadas em Arch Linux
\end{minted}

\subsection{Instalação do \emph{Create React App}}
\begin{minted}[autogobble,breaklines,linenos,frame=lines,fontsize=\footnotesize]{bash}
	sudo npm install -g create-react-app
\end{minted}

\section{Windows}
\subsection{Instalação do Node}
Pode ser baixado através do \emph{site} oficial:
\href{https://nodejs.org/en/download/}{https://nodejs.org/en/download/}

\subsection{Instalação do \textsc{npm}}
Como o NPM vem do Node e é instalado por padrão, não é preciso instalá-lo.

\subsection{Instalação do \emph{Create React App}}
\begin{minted}[autogobble,breaklines,linenos,frame=lines,fontsize=\footnotesize]{bash}
	npm install -g create-react-app
\end{minted}

\section{Instalação do \emph{CouchDB}}
Cada distribuição linux tem a instalação do \emph{CouchDB} de uma forma, podendo também baixar direto do \href{https://couchdb.apache.org}{site}, assim como feito no Windows. No GNU/Linux é importante verificar se o serviço/\emph{daemon} do banco de dados está funcionando, não colocaremos código para exemplificar, pois pode mudar de distribuição para distribuição.

Para verificar se o banco de dados está funcionando, basta entrar no endereço \href{http://localhost:5984/}{http://localhost:5984/}.

Para o funcionamento da aplicação, precisamos ativar as permossões de acesso de domínios, para acessar a \textsc{api} do banco basta acessar \href{http://localhost:5984/\_utils}{http://localhost:5984/\_utils}, com isso basta ir nas configurações do \emph{CouchDB}, na opção \textsc{cors} e clicar em \emph{Enable \textsc{cors}} e selecionar a opção \emph{All domains}.

\subsection{Alimentando o banco de dados}
Para colocar os conteúdos que temos no banco de dados, temos que utilizar os arquivos contidos na pasta \texttt{database/} da raiz do projeto, então vamos para a pasta:

\begin{minted}[autogobble,breaklines,linenos,frame=lines,fontsize=\footnotesize]{bash}
	cd database
\end{minted}

Antes de adicionar o conteúdo dos \texttt{.json}, precisamos criar o banco de dados para depois inserir as tuplas, então precisamos rodar os seguintes comandos:

\begin{minted}[autogobble,breaklines,linenos,frame=lines,fontsize=\footnotesize]{bash}
	curl -X PUT http://localhost:5984/petshop
	curl -X PUT http://localhost:5984/pets
	curl -X PUT http://localhost:5984/usuarios
	curl -X PUT http://localhost:5984/servicos
\end{minted}

E, agora para inserir os dados no banco de dados:

\begin{minted}[autogobble,breaklines,linenos,frame=lines,fontsize=\footnotesize]{bash}
	curl -d @petshop.json -H "Content-type: application/json" -X POST http://localhost:5984/petshop/_bulk_docs
	curl -d @pets.json -H "Content-type: application/json" -X POST http://localhost:5984/pets/_bulk_docs
	curl -d @usuarios.json -H "Content-type: application/json" -X POST http://localhost:5984/usuarios/_bulk_docs
	curl -d @servicos.json -H "Content-type: application/json" -X POST http://localhost:5984/servicos/_bulk_docs
\end{minted}

Com estes comandos temos todas as ferramentas necesssárias para o funcionamento
da nossa aplicação, tanto no Windows como no Linux.

\section{Rodando a aplicação}
Agora precisamos que entre na pasta do projeto pelo terminal.

\begin{minted}[autogobble,breaklines,linenos,frame=lines,fontsize=\footnotesize]{bash}
	cd ReactPetShop #Tanto Windows quanto Linux
\end{minted}

Rode o projeto instalando suas dependências:

\begin{minted}[autogobble,breaklines,linenos,frame=lines,fontsize=\footnotesize]{bash}
	npm install && npm start      #Windows
	npm install && npm start      #Linux
\end{minted}

Observação: caso, não consiga rodar o comando \texttt{npm install}, exclua a pasta \texttt{node\_modules} da raiz do projeto e tente novamente os comandos anteriores.

O projeto é aberto no navegador padrão assim que tudo estiver pronto, caso isso
não ocorra, é possível acessá-lo através do \emph{link}:
\href{http://localhost:3000/}{http://localhost:3000/}



























\chapter{Funcionalidades}
\section{Página de \emph{index}}
Na página inicial o logo da aplicação sendo exibido corretamente e, de forma global, redireciona o usuário para o \emph{index} ao clicar.

A \emph{navbar} está funcional funcional, sendo possível acessar qualquer categoria desse menu. Na área de usuário, quando o usuário está “logado”, as opções de perfil de usuário, carrinho e sair são acessíveis. Já na barra de busca, temos uma busca funcional para buscas exatas, conforme descrito no item~\ref{sub:barra}.

Todos produtos, de todas as categorias, estão sendo exibidos com suas respectivas informações e o produto listado é acessível para área do produto, mostrando suas devidas informações.

\section{Página de produto}
Exibe o produto selecionado corretamente, permite a mudança das imagens armazenadas no banco e exibe suas informações contidas no banco.

O botão comprar adiciona o produto no carrinho e já redireciona o usuário para a página do carrinho, se tiver feito o \emph{login}, caso não tiver com a sua conta “logada”, ele redireciona para o carrinho, mas avisa que precisa estar “logado” para vizualizá-lo. Obs: ao logar, seu carrinho será exibido com o produto selecionado previamente.

\section{Página do carrinho}
A página exibe os produtos que estão no carrinho, se caso tiver produtos. Permite mudar a quantidade de cada produto, permite exclusão dos produtos do carrinho, porém quando uma exclusão é feita, não é feito a atualização automática da área do carrinho.

O botão fechar carrinho, finaliza uma compra, removendo todos os produtos do carrinho e redireciona o usuário para  a página principal. No entanto a compra não é salva em banco.

\section{Página de serviço}
Todos os serviços são exibidos corretamente com todas as suas informações trazidas do banco e cada serviço tem o seu botão agendar funcionando que, ao clicar, redireciona o usuário para a página de agendamento do serviço selecionado, se caso o usuário estiver “logado”.

\section{Página de agendamento}
O serviço que foi selecionado previamente é exibido com todas as suas informações. O usuário “logado” é reconhecido e seus \emph{pets} são listados, permitindo seleção do \emph{pet} para agendamento.

A escolha de data e hora está funcional, porém sem restrições. Já o valor do serviço é exibido, porém o agendamento não é registrado em banco ao clicar em agendar e o usuário é redirecionado para a página inicial.

\section{Página de busca}
A página de busca é acessada através de uma busca realizada na barra de busca, mostrando os produtos que foram encontrados pela ação realizada. Os produtos são exibidos corretamente.

Um subtítulo da página exbibe corretamente o que foi buscado e todos os produtos com este nome são exibidos e podendo ser acessados clicando em sua miniatura.

\section{Página de usuário}
Ao entrar com seu usuário e senha, a área de \emph{login} é modificada, permitindo o acesso ao perfil do usuário. As informações de usuário são buscadas e exibidas corretamente, com todos os seus \emph{pets}, que são listados, podendo ser selecionado para acessar suas informações no perfil do \emph{pet} escolhido.

Caso este usuário seja um administrador, uma área do administrador é exibida, caso contrário apenas são listadas áreas de compras e serviços realizados. Se o usuário for um administrador, ele pode acessar as áreas de inserção de produtos, serviços e usuários.

Compras e serviços realizados não são exibidos, assim como a área do administrador não é funcional. Ao clicar no botão de adicionar \emph{pet}, o usuário é redirecionado para um formulário de cadastro do \emph{pet}.

\section{Páginas de cadastro}
As páginas de cadastro de produtos, usuários e serviços são acessadas somente por um administrador, já o cadastro de \emph{pets} pode ser feito por qualquer tipo de usuário.

As páginas de cadastro estão com todos os campos funcionais, porém o botão cadastrar não tem funcionalidade.
























% ----------------------------------------------------------

% ----------------------------------------------------------
% ELEMENTOS PÓS-TEXTUAIS
% ----------------------------------------------------------
\postextual
% ----------------------------------------------------------

% ----------------------------------------------------------
% Referências bibliográficas
% ----------------------------------------------------------
%\bibliography{trabalho}

\end{document}
