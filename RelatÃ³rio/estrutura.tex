\chapter{Estrutura do \emph{site}}
Como o nosso \emph{site} é baseado praticamente em componentização, neste
capítulo passaremos uma noção de como o estruturamos.

\section{\emph{APP}}
No começo de tudo, temos a nossa aplicação, mais conhecida como \emph{APP},
componente a qual consiste o nosso \emph{site} inteiro e que será responsável
por formar nosso único arquivo \texttt{.html}, chamado \texttt{index.html}. A
\emph{APP} está separada em 4 partes, as quais são:
	\begin{description}[style=nextline]
		\item[Topo] Componente que contém o logo do PetShop, uma barra de
pesquisa e a parte de \emph{login} ou, se já estiver “\emph{logado}”, área do
usuário.
		\item[\emph{Navbar}] Um menu que serve para acessar de forma mais
intuitiva algumas partes do \emph{site} como \textbf{Início}, o qual você
consegue acessar de qualquer lugar do \emph{site} a página inicial,
\textbf{Serviços}, onde você consegue de qualquer lugar da aplicação ir para a
página de serviços que estão disponíveis, e alguns tipos de animais, como
\textbf{Cachorros}, \textbf{Gatos}, entre outros, que são um atalho para os
produtos da respectiva espécie.
		\item[Conteúdo] Além da parte do \textbf{Topo}, a qual tem a área de
\emph{login}, essa é a única parte do nosso \emph{site} que se muda. Na tela
incial, no componente \textbf{Conteúdo} temos uma área que fica um \emph{slide}
de imagens mostrando os destaques que o dono da aplicação pode colocar. Logo
abaixo temos listado alguns produtos de cada espécie, esses produtos, na nossa
aplicação está pegando os quatro primeiros de cada espécie que estão
cadastrados. Nada impede que coloquemos os produtos de cada espécie que estão em
promoção.
		\item[Rodapé] No rodapé, há o nome completo de todos os integrantes do
grupo, onde cada nome é um \emph{link} que redireciona o usuário para o a conta
do GitHub de cada um.
	\end{description}

\subsection{Topo}
Como foi dito anteriormente, no componente \textbf{Topo} nós temos três itens
basicamente, o logo da aplicação, o qual também é uma ligação para a página
inicial, uma barra de busca no \emph{site} e uma área de acesso ao usuário.

\subsubsection{Barra de busca}

\subsubsection{\emph{Login}}
\subsection{\emph{Navbar}}
\subsection{Conteúdo}
\subsection{Rodapé}
