\chapter*[Introdução]{Introdução}
Nesta terceira fase, a  final, da aplicação PetShop, nos foi solicitado que
concluíssemos o desenvolvimento de uma página \emph{web} utilizando as
linguagens \textsc{html5}, \textsc{css3}, as quais havíamos utilizado para
desenvolver a primeira parte do projeto, JavaScript e o uso de \emph{frameworks}, vistos na segunda parte, e agora a utilização de um banco de dados.

O objetivo desta parte é o refinamento de uma página no estilo
\emph{Single-Page Application} de um PetShop. Por termos feito já o
\emph{mockup} na primeira parte do trabalho e a utilização de JavaScrip com \emph{frameworks} da segunda parte, acreditamos que será mais simples de fazer essa terceira, e última, parte.

Na segunda parte do trabalho, utilizamos uma \textsc{api} para simular um banco de dados, pois pensamos que seria mais simples a migração para um banco que utiliza a arquitetura \textsc{rest}ful. Escolhemos para essa parte o \emph{CouchDB}, um banco de dados no\textsc{sql} que utiliza como linguagem JavaScrip e protocolo de comunicação o \textsc{http}.
