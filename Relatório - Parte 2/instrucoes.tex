\chapter{Instruções}
Neste capítulo vamos mostrar como utilizar a nossa aplicação, podendo rodar
tanto \emph{online} como \emph{offline}. Os comandos estão listados em ordem,
portanto a execução deles resulta no funcionamento do projeto. Na sequência,
teremos a instalação das ferramentas necessárias no Linux e no Windows e
posteriormente como será o procedimento para rodar a aplicação.

\section{Linux}
\subsection{Instalação do Node}
\begin{minted}[autogobble,breaklines,linenos,frame=lines,fontsize=\footnotesize]{bash}
	sudo apt-get install nodejs #Distros baseadas em Debian ou
	sudo pacman -S nodejs       #Distros baseadas em Arch Linux
\end{minted}

\subsection{Instalação do \textsc{npm}}
\begin{minted}[autogobble,breaklines,linenos,frame=lines,fontsize=\footnotesize]{bash}
	sudo apt-get install npm #Distros baseadas em Debian ou
	sudo pacman -S npm       #Distros baseadas em Arch Linux
\end{minted}

\subsection{Instalação do \emph{Create React App}}
\begin{minted}[autogobble,breaklines,linenos,frame=lines,fontsize=\footnotesize]{bash}
	sudo npm install -g create-react-app
\end{minted}

\section{Windows}
\subsection{Instalação do Node}
Pode ser baixado através do \emph{site} oficial:
\href{https://nodejs.org/en/download/}{https://nodejs.org/en/download/}

\subsection{Instalação do \textsc{npm}}
Como o NPM vem no pacode do Node e é instalado por padrão, não é preciso
instalá-lo.

\subsection{Instalação do \emph{Create React App}}
\begin{minted}[autogobble,breaklines,linenos,frame=lines,fontsize=\footnotesize]{bash}
	npm install -g create-react-app
\end{minted}

Com estes comandos temos todas as ferramentas necesssárias para o funcionamento
da nossa aplicação, tanto no Windows como no Linux.

\section{Rodando a aplicação}
Agora precisamos que entre na pasta do projeto pelo terminal.

\begin{minted}[autogobble,breaklines,linenos,frame=lines,fontsize=\footnotesize]{bash}
	cd ReactPetShop #Tanto Windows quanto Linux
\end{minted}

Rode o projeto instalando suas dependências:
\begin{minted}[autogobble,breaklines,linenos,frame=lines,fontsize=\footnotesize]{bash}
	npm install && npm start      #Windows
	sudo npm install && npm start #Linux
\end{minted}

O projeto é aberto no navegador padrão assim que tudo estiver pronto, caso isso
não ocorra, é possível acessá-lo através do \emph{link}:
\href{http://localhost:3000/}{http://localhost:3000/}

\section{Opção \emph{offline}}
\label{sec:offline}
Caso queira rodar a \textsc{api} \emph{offline} siga estes comandos, mas nenhum
deles é necessário para o funcionamento do projeto. Mantenha o terminal do
processo passado e utilize outra aba/janela do terminal para realizar as
próximas ações.

\subsection{Instalação do Json Server}
\begin{minted}[autogobble,breaklines,linenos,frame=lines,fontsize=\footnotesize]{bash}
	npm install -g json-server      #Windows
	sudo npm install -g json-server #Linux
\end{minted}

Vá até a pasta do projeto.

\begin{minted}[autogobble,breaklines,linenos,frame=lines,fontsize=\footnotesize]{bash}
	cd ReactPetShop #Tanto Windows quanto Linux
\end{minted}

\subsection{Rodando a \textsc{api} \emph{offline}}
\begin{minted}[autogobble,breaklines,linenos,frame=lines,fontsize=\footnotesize]{bash}
	json-server --watch --port 3001 db.json #Tanto Windows quanto Linux
\end{minted}

Na pasta do projeto, abra o arquivo \texttt{src > service > http.js}. Nesse
arquivo existem duas linhas de código:

\begin{minted}[autogobble,breaklines,linenos,frame=lines,fontsize=\footnotesize]{javascript}
	baseURL: 'https://my-json-server.typicode.com/ederrr/ReactPetShop/'
	//baseURL: 'http://localhost:3001'
\end{minted}

Comente a primeira linha e descomente a segunda linha, ficando da seguinte
forma:

\begin{minted}[autogobble,breaklines,linenos,frame=lines,fontsize=\footnotesize]{javascript}
	//baseURL: 'https://my-json-server.typicode.com/ederrr/ReactPetShop/'
	baseURL: 'http://localhost:3001'
\end{minted}

Pronto, seu servidor \emph{offline} está rodando a \textsc{api} e seu projeto
está consumindo-a.



























