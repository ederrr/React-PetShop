%%%%%%%%%%%%%%%%%%%%%%%%%%%%%%%%%%%%%%%%%%%%%%%%%%%%%%%%%%%
%			Projeto 2 - React Petshop
%
%	Introdução ao Desenvolvimento Web			SCC0219
%	Prof. Dilvan de Abreu Moreira
%
%	Eder Rosati Ribeiro							8122585
%	Leonardo de Almeida Lima Zanguetin			8531866
%	Victor Luiz da Silva Mariano Pereira		8602444
%
%%%%%%%%%%%%%%%%%%%%%%%%%%%%%%%%%%%%%%%%%%%%%%%%%%%%%%%%%%%

% ---
% Definição do documento
% ---
\documentclass[
	% -- opções da classe memoir --
	12pt,				% tamanho da fonte
	openright,			% capítulos começam em pág ímpar
						% (insere página vazia caso preciso)
	%openany,			% capítulos começam em qualquer página
	twoside,			% para impressão em recto e verso. Oposto a oneside
	%oneside,			% para impressão somente em recto
	a4paper,			% tamanho do papel.
]{abntex2}
% ---

% ---
% Pacotes básicos
% ---
\usepackage{fontspec}
\usepackage{polyglossia}
	\setdefaultlanguage{portuges}
\usepackage{lmodern}			% Usa a fonte Latin Modern
\usepackage{indentfirst}		% Indenta o primeiro parágrafo de cada seção.
\usepackage{color}				% Controle das cores
\usepackage{graphicx}			% Inclusão de gráficos
	\graphicspath{{Figuras/}}
\usepackage{microtype}			% para melhorias de justificação
\usepackage{amssymb}
% ---

% ---
% Pacotes de citações
% ---
\usepackage[brazilian,hyperpageref]{backref}	% Paginas com as citações na
												% bibliografia
\usepackage[alf]{abntex2cite}	% Citações padrão ABNT
\usepackage{caption}			% Pacote para resolver o problema do
								% "listings", "minted" e códigos muito grandes
\usepackage{listings}			% Pacote para listagens de códigos
\usepackage{minted}				% Inserção de códigos fonte
% ---

% ---
% Pacotes para tabelas
% ---
\usepackage{longtable}
\usepackage{booktabs}
\usepackage{array}
% ---

% ---
% Pacotes adicionais
% ---
%\usepackage{showframe}			% Pacote para mostrar as caixas de texto
% ---

% ---
% Informações de dados para CAPA e FOLHA DE ROSTO
% ---
\titulo{Projeto 2 \\ \emph{React} PetShop}
\autor{
		Eder Rosati Ribeiro						---	8122585 \and\\
		Leonardo de Almeida Lima Zanguetin		---	8531866	\and\\
		Victor Luiz da Silva Mariano Pereira	---	8602444 \\
}
\local{Brasil}
\data{2018}
\instituicao{%
  Universidade de São Paulo -- USP
  \par
  Instituto de Ciências Matemáticas e de Computação -- ICMC
  \par
  Introdução ao Desenvolvimento Web -- SCC0219}
\tipotrabalho{Trabalho Acadêmico}
% ---

% ---
% Configurações de aparência do PDF final

% alterando o aspecto da cor azul
\definecolor{blue}{RGB}{41,5,195}

% informações do PDF
\makeatletter
\hypersetup{
	%pagebackref=true,
	pdftitle={\@title},
	pdfauthor={\@author},
	pdfsubject={React PetShop},
	pdfcreator={LuaLaTeX with abnTeX2},
	pdfkeywords={USP }{ICMC }{Introdução ao Desenvolvimento Web }{React }{PetShop},
	colorlinks=true,			% false: boxed links; true: colored links
	linkcolor=blue,				% color of internal links
	citecolor=blue,				% color of links to bibliography
	filecolor=magenta,			% color of file links
	urlcolor=blue,
	bookmarksdepth=4
}
\makeatother
% ---

% ---
% Espaçamentos entre linhas e parágrafos
% ---

% O tamanho do parágrafo é dado por:
\setlength{\parindent}{1.3cm}

% Controle do espaçamento entre um parágrafo e outro:
\setlength{\parskip}{0.2cm}  % tente também \onelineskip

% ---
% compila o indice
% ---
\makeindex
% ---

% ----
% Início do documento
% ----
\begin{document}

% Retira espaço extra obsoleto entre as frases.
\frenchspacing

% ----------------------------------------------------------
% ELEMENTOS PRÉ-TEXTUAIS
% ----------------------------------------------------------
\pretextual

% ---
% Capa
% ---
\imprimircapa
% ---

% ---
% Folha de rosto
% (o * indica que haverá a ficha bibliográfica)
% ---
\imprimirfolhaderosto*
% ---

% ---
% RESUMOS
% ---
% resumo em português
%\setlength{\absparsep}{18pt} % ajusta o espaçamento dos parágrafos do resumo
%\begin{resumo}
%\end{resumo}

% ---
% inserir lista de ilustrações
% ---
%\pdfbookmark[0]{\listfigurename}{lof}
%\listoffigures*
%\clearpage
% ---

% ---
% inserir lista de tabelas
% ---
%\pdfbookmark[0]{\listtablename}{lot}
%\listoftables*
%\clearpage
% ---

% ---
% inserir lista de códigos fonte
% ---
%\renewcommand\listoflistingscaption{Lista de Códigos Fonte}
%\listoflistings
%\clearpage
% ---

% ---
% inserir o sumario
% ---
%\pdfbookmark[0]{\contentsname}{toc}
%\tableofcontents*
%\clearpage
% ---

% ----------------------------------------------------------
% ELEMENTOS TEXTUAIS
% ----------------------------------------------------------
\textual

% ----------------------------------------------------------
% Corpo do trabalho
% ----------------------------------------------------------
\chapter*[Introdução]{Introdução}
Nesta terceira fase, a  final, da aplicação PetShop, nos foi solicitado que
concluíssemos o desenvolvimento de uma página \emph{web} utilizando as
linguagens \textsc{html5}, \textsc{css3}, as quais havíamos utilizado para
desenvolver a primeira parte do projeto, JavaScript e o uso de \emph{frameworks}, vistos na segunda parte, e agora a utilização de um banco de dados.

O objetivo desta parte é o refinamento de uma página no estilo
\emph{Single-Page Application} de um PetShop. Por termos feito já o
\emph{mockup} na primeira parte do trabalho e a utilização de JavaScrip com \emph{frameworks} da segunda parte, acreditamos que será mais simples de fazer essa terceira, e última, parte.

Na segunda parte do trabalho, utilizamos uma \textsc{api} para simular um banco de dados, pois pensamos que seria mais simples a migração para um banco que utiliza a arquitetura \textsc{rest}ful. Escolhemos para essa parte o \emph{CouchDB}, um banco de dados no\textsc{sql} que utiliza como linguagem JavaScrip e protocolo de comunicação o \textsc{http}.

\chapter{Funcionamento}
Como foi dito anteriormente, todo o \emph{site} foi desenvolvido utilizando o
React, uma biblioteca responsável apenas pelo desenvolvimento do
\emph{front-end} em \emph{Single Page Application} (\textsc{spa}). Tudo que
julgamos conveniente à reutilização foi componentizado para reaproveitamento,
tanto nos itens que precisavam ser reutilizáveis quanto nas partes que tinham
papéis diferentes dentro do \emph{site}.

Essa biblioteca JavaScript roda através do Node, portanto é necessário a
instalação do Node e do \textsc{npm} (do inglês, \emph{Node Package Manager})
para iniciar um projeto. Com o \textsc{npm} é possível baixar e instalar o
\emph{Create React App}, um pacote responsável por configurar várias tecnologias
e trazer um projeto pronto para o desenvolvimento com todas as ferramentas
iniciais mais utilizadas no React. Uma dessas tecnologias é o \textsc{jsx},
capaz de transformar os códigos JavaScript em códigos parecidos com
\textsc{html5}.

Pode haver estranhamento com o código desenvolvido por aparentar utilizar
\textsc{html} misturado com JavaScript, porém, é tudo JavaScript e o Babel nos
auxilia nessa etapa. Todo o código pode ser realizado em ECMAScrip 6 pois é
transpilado e convertido em ECMAScript 5 para funcionar nos navegadores atuais
que ainda não o suportam.

O \emph{site} todo consome uma \textsc{api} que hospedamos no GitHub servindo
como interface entre nosso \emph{front-end} e \emph{back-end}. O projeto já está
preparado para consumir esta \textsc{api} hospedada \emph{online}, mas caso
queira que o projeto rode totalmente \emph{offline}, deixamos preparado para
isso. Na seção~\ref{sec:offline} será abordada como podermemos rodar \emph{offline}
utilizando Json Server.


\chapter{Estrutura do \emph{site}}
Como o nosso \emph{site} é baseado praticamente em componentização, neste
capítulo passaremos uma noção de como o estruturamos.

\section{\emph{APP}}
No começo de tudo, temos a nossa aplicação, mais conhecida como \emph{APP},
componente a qual consiste o nosso \emph{site} inteiro e que será responsável
por formar nosso único arquivo \texttt{.html}, chamado \texttt{index.html}. A
\emph{APP} está separada em 4 partes, as quais são:
	\begin{description}[style=nextline]
		\item[Topo] Componente que contém o logo do PetShop, uma barra de
pesquisa e a parte de \emph{login} ou, se já estiver “\emph{logado}”, área do
usuário.
		\item[\emph{Navbar}] Um menu que serve para acessar de forma mais
intuitiva algumas partes do \emph{site} como \textbf{Início}, o qual você
consegue acessar de qualquer lugar do \emph{site} a página inicial,
\textbf{Serviços}, onde você consegue de qualquer lugar da aplicação ir para a
página de serviços que estão disponíveis, e alguns tipos de animais, como
\textbf{Cachorros}, \textbf{Gatos}, entre outros, que são um atalho para os
produtos da respectiva espécie.
		\item[Conteúdo] Além da parte do \textbf{Topo}, a qual tem a área de
\emph{login}, essa é a única parte do nosso \emph{site} que se muda. Na tela
incial, no componente \textbf{Conteúdo} temos uma área que fica um \emph{slide}
de imagens mostrando os destaques que o dono da aplicação pode colocar. Logo
abaixo temos listado alguns produtos de cada espécie, esses produtos, na nossa
aplicação está pegando os quatro primeiros de cada espécie que estão
cadastrados. Nada impede que coloquemos os produtos de cada espécie que estão em
promoção.
		\item[Rodapé] No rodapé, há o nome completo de todos os integrantes do
grupo, onde cada nome é um \emph{link} que redireciona o usuário para o a conta
do GitHub de cada um.
	\end{description}

\subsection{Topo}
Como foi dito anteriormente, no componente \textbf{Topo} nós temos três itens
basicamente, o logo da aplicação, o qual também é uma ligação para a página
inicial, uma barra de busca no \emph{site} e uma área de acesso ao usuário.
\emph{login}.
\subsubsection{Barra de busca}

\subsubsection{Login}
\subsection{\emph{Navbar}}
\subsection{Conteúdo}
\subsection{Rodapé}

\chapter{Instruções}
Neste capítulo vamos mostrar como utilizar a nossa aplicação, os comandos estão listados em ordem, portanto a execução deles resulta no funcionamento do projeto. Na sequência, teremos a instalação das ferramentas necessárias no Linux e no Windows e posteriormente como será o procedimento para rodar a aplicação.

\section{Linux}
\subsection{Instalação do Node}
\begin{minted}[autogobble,breaklines,linenos,frame=lines,fontsize=\footnotesize]{bash}
	sudo apt-get install nodejs #Distros baseadas em Debian ou
	sudo pacman -S nodejs       #Distros baseadas em Arch Linux
\end{minted}

\subsection{Instalação do \textsc{npm}}
\begin{minted}[autogobble,breaklines,linenos,frame=lines,fontsize=\footnotesize]{bash}
	sudo apt-get install npm #Distros baseadas em Debian ou
	sudo pacman -S npm       #Distros baseadas em Arch Linux
\end{minted}

\subsection{Instalação do \emph{Create React App}}
\begin{minted}[autogobble,breaklines,linenos,frame=lines,fontsize=\footnotesize]{bash}
	sudo npm install -g create-react-app
\end{minted}

\section{Windows}
\subsection{Instalação do Node}
Pode ser baixado através do \emph{site} oficial:
\href{https://nodejs.org/en/download/}{https://nodejs.org/en/download/}

\subsection{Instalação do \textsc{npm}}
Como o NPM vem do Node e é instalado por padrão, não é preciso instalá-lo.

\subsection{Instalação do \emph{Create React App}}
\begin{minted}[autogobble,breaklines,linenos,frame=lines,fontsize=\footnotesize]{bash}
	npm install -g create-react-app
\end{minted}

\section{Instalação do \emph{CouchDB}}
Cada distribuição linux tem a instalação do \emph{CouchDB} de uma forma, podendo também baixar direto do \href{https://couchdb.apache.org}{site}, assim como feito no Windows. No GNU/Linux é importante verificar se o serviço/\emph{daemon} do banco de dados está funcionando, não colocaremos código para exemplificar, pois pode mudar de distribuição para distribuição.

Para verificar se o banco de dados está funcionando, basta entrar no endereço \href{http://localhost:5984/}{http://localhost:5984/}.

Para o funcionamento da aplicação, precisamos ativar as permossões de acesso de domínios, para acessar a \textsc{api} do banco basta acessar \href{http://localhost:5984/\_utils}{http://localhost:5984/\_utils}, com isso basta ir nas configurações do \emph{CouchDB}, na opção \textsc{cors} e clicar em \emph{Enable \textsc{cors}} e selecionar a opção \emph{All domains}.

\subsection{Alimentando o banco de dados}
Para colocar os conteúdos que temos no banco de dados, temos que utilizar os arquivos contidos na pasta \texttt{database/} da raiz do projeto, então vamos para a pasta:

\begin{minted}[autogobble,breaklines,linenos,frame=lines,fontsize=\footnotesize]{bash}
	cd database
\end{minted}

Antes de adicionar o conteúdo dos \texttt{.json}, precisamos criar o banco de dados para depois inserir as tuplas, então precisamos rodar os seguintes comandos:

\begin{minted}[autogobble,breaklines,linenos,frame=lines,fontsize=\footnotesize]{bash}
	curl -X PUT http://localhost:5984/petshop
	curl -X PUT http://localhost:5984/pets
	curl -X PUT http://localhost:5984/usuarios
	curl -X PUT http://localhost:5984/servicos
\end{minted}

E, agora para inserir os dados no banco de dados:

\begin{minted}[autogobble,breaklines,linenos,frame=lines,fontsize=\footnotesize]{bash}
	curl -d @petshop.json -H "Content-type: application/json" -X POST http://localhost:5984/petshop/_bulk_docs
	curl -d @pets.json -H "Content-type: application/json" -X POST http://localhost:5984/pets/_bulk_docs
	curl -d @usuarios.json -H "Content-type: application/json" -X POST http://localhost:5984/usuarios/_bulk_docs
	curl -d @servicos.json -H "Content-type: application/json" -X POST http://localhost:5984/servicos/_bulk_docs
\end{minted}

Com estes comandos temos todas as ferramentas necesssárias para o funcionamento
da nossa aplicação, tanto no Windows como no Linux.

\section{Rodando a aplicação}
Agora precisamos que entre na pasta do projeto pelo terminal.

\begin{minted}[autogobble,breaklines,linenos,frame=lines,fontsize=\footnotesize]{bash}
	cd ReactPetShop #Tanto Windows quanto Linux
\end{minted}

Rode o projeto instalando suas dependências:

\begin{minted}[autogobble,breaklines,linenos,frame=lines,fontsize=\footnotesize]{bash}
	npm install && npm start      #Windows
	npm install && npm start      #Linux
\end{minted}

Observação: caso, não consiga rodar o comando \texttt{npm install}, exclua a pasta \texttt{node\_modules} da raiz do projeto e tente novamente os comandos anteriores.

O projeto é aberto no navegador padrão assim que tudo estiver pronto, caso isso
não ocorra, é possível acessá-lo através do \emph{link}:
\href{http://localhost:3000/}{http://localhost:3000/}



























% ----------------------------------------------------------

% ----------------------------------------------------------
% ELEMENTOS PÓS-TEXTUAIS
% ----------------------------------------------------------
\postextual
% ----------------------------------------------------------

% ----------------------------------------------------------
% Referências bibliográficas
% ----------------------------------------------------------
%\bibliography{trabalho}

\end{document}
