\chapter{Funcionamento}
Como foi dito anteriormente, todo o \emph{site} foi desenvolvido utilizando o
React, uma biblioteca responsável apenas pelo desenvolvimento do
\emph{front-end} em \emph{Single Page Application} (\textsc{spa}). Tudo que
julgamos conveniente à reutilização foi componentizado para reaproveitamento,
tanto nos itens que precisavam ser reutilizáveis quanto nas partes que tinham
papéis diferentes dentro do \emph{site}.

Essa biblioteca JavaScript roda através do Node, portanto é necessário a
instalação do Node e do \textsc{npm} (do inglês, \emph{Node Package Manager})
para iniciar um projeto. Com o \textsc{npm} é possível baixar e instalar o
\emph{Create React App}, um pacote responsável por configurar várias tecnologias
e trazer um projeto pronto para o desenvolvimento com todas as ferramentas
iniciais mais utilizadas no React. Uma dessas tecnologias é o \textsc{jsx},
capaz de transformar os códigos JavaScript em códigos parecidos com
\textsc{html5}.

Pode haver estranhamento com o código desenvolvido por aparentar utilizar
\textsc{html} misturado com JavaScript, porém, é tudo JavaScript e o Babel nos
auxilia nessa etapa. Todo o código pode ser realizado em ECMAScrip 6 pois é
transpilado e convertido em ECMAScript 5 para funcionar nos navegadores atuais
que ainda não o suportam.

O \emph{site} todo consome uma \textsc{api} que hospedamos no GitHub servindo
como interface entre nosso \emph{front-end} e \emph{back-end}. O projeto já está
preparado para consumir esta \textsc{api} hospedada \emph{online}, mas caso
queira que o projeto rode totalmente \emph{offline}, deixamos preparado para
isso. Na seção~\ref{sec:offline} será abordada como podermemos rodar \emph{offline}
utilizando Json Server.

