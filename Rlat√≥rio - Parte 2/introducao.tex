\chapter{Introdução}

Nesta segunda fase da aplicação PetShop, nos foi solicitado que
continuássemos com o desenvolvimento de uma página \emph{web} utilizando as
linguagens \textsc{html5}, \textsc{css3}, as quais havíamos utilizado para
desenvolver a primeira parte do projeto, e, as novidades, JavaScript e o uso de
\emph{frameworks}.

O objetivo deste trabalho ainda é o desenvolvimento de uma página no estilo
\emph{Single-Page Application} de um PetShop. Por termos feito já o
\emph{mockup} na primeira parte do trabalho e com a utilização dessa nova
linguagem, acreditamos que será mais simples de fazer essa segunda parte.

O \textsc{html5}, como sabemos, cuida da parte estrutural do nosso \emph{site},
o \textsc{css3}, pelo estilo da página. Já o JavaScript é responsável pelo
comportamento da nossa página, ou seja, tudo que acontecer na página sem
precisar atualizá-la, é parte do JavaScript.

Para facilitar o desenvolvimento nessa nova linguagem, escolhemos o React, uma
biblioteca declarativa de JavaScript que serve para criar interfaces visuais.
Com ela, podemos motrar mais dinamismo utilizado React, pois ele além de ser
reativo, ou seja, ao mudar o estado de um componente o que ele representa muda
também, podemos economizar tempo e minimizar em linhas de código, porque
podemos usar componentes reutilizáveis.
